% Options for packages loaded elsewhere
\PassOptionsToPackage{unicode}{hyperref}
\PassOptionsToPackage{hyphens}{url}
%
\documentclass[
]{article}
\usepackage{lmodern}
\usepackage{amssymb,amsmath}
\usepackage{ifxetex,ifluatex}
\ifnum 0\ifxetex 1\fi\ifluatex 1\fi=0 % if pdftex
  \usepackage[T1]{fontenc}
  \usepackage[utf8]{inputenc}
  \usepackage{textcomp} % provide euro and other symbols
\else % if luatex or xetex
  \usepackage{unicode-math}
  \defaultfontfeatures{Scale=MatchLowercase}
  \defaultfontfeatures[\rmfamily]{Ligatures=TeX,Scale=1}
\fi
% Use upquote if available, for straight quotes in verbatim environments
\IfFileExists{upquote.sty}{\usepackage{upquote}}{}
\IfFileExists{microtype.sty}{% use microtype if available
  \usepackage[]{microtype}
  \UseMicrotypeSet[protrusion]{basicmath} % disable protrusion for tt fonts
}{}
\makeatletter
\@ifundefined{KOMAClassName}{% if non-KOMA class
  \IfFileExists{parskip.sty}{%
    \usepackage{parskip}
  }{% else
    \setlength{\parindent}{0pt}
    \setlength{\parskip}{6pt plus 2pt minus 1pt}}
}{% if KOMA class
  \KOMAoptions{parskip=half}}
\makeatother
\usepackage{xcolor}
\IfFileExists{xurl.sty}{\usepackage{xurl}}{} % add URL line breaks if available
\IfFileExists{bookmark.sty}{\usepackage{bookmark}}{\usepackage{hyperref}}
\hypersetup{
  pdftitle={ELSSP},
  hidelinks,
  pdfcreator={LaTeX via pandoc}}
\urlstyle{same} % disable monospaced font for URLs
\usepackage[margin=1in]{geometry}
\usepackage{color}
\usepackage{fancyvrb}
\newcommand{\VerbBar}{|}
\newcommand{\VERB}{\Verb[commandchars=\\\{\}]}
\DefineVerbatimEnvironment{Highlighting}{Verbatim}{commandchars=\\\{\}}
% Add ',fontsize=\small' for more characters per line
\usepackage{framed}
\definecolor{shadecolor}{RGB}{248,248,248}
\newenvironment{Shaded}{\begin{snugshade}}{\end{snugshade}}
\newcommand{\AlertTok}[1]{\textcolor[rgb]{0.94,0.16,0.16}{#1}}
\newcommand{\AnnotationTok}[1]{\textcolor[rgb]{0.56,0.35,0.01}{\textbf{\textit{#1}}}}
\newcommand{\AttributeTok}[1]{\textcolor[rgb]{0.77,0.63,0.00}{#1}}
\newcommand{\BaseNTok}[1]{\textcolor[rgb]{0.00,0.00,0.81}{#1}}
\newcommand{\BuiltInTok}[1]{#1}
\newcommand{\CharTok}[1]{\textcolor[rgb]{0.31,0.60,0.02}{#1}}
\newcommand{\CommentTok}[1]{\textcolor[rgb]{0.56,0.35,0.01}{\textit{#1}}}
\newcommand{\CommentVarTok}[1]{\textcolor[rgb]{0.56,0.35,0.01}{\textbf{\textit{#1}}}}
\newcommand{\ConstantTok}[1]{\textcolor[rgb]{0.00,0.00,0.00}{#1}}
\newcommand{\ControlFlowTok}[1]{\textcolor[rgb]{0.13,0.29,0.53}{\textbf{#1}}}
\newcommand{\DataTypeTok}[1]{\textcolor[rgb]{0.13,0.29,0.53}{#1}}
\newcommand{\DecValTok}[1]{\textcolor[rgb]{0.00,0.00,0.81}{#1}}
\newcommand{\DocumentationTok}[1]{\textcolor[rgb]{0.56,0.35,0.01}{\textbf{\textit{#1}}}}
\newcommand{\ErrorTok}[1]{\textcolor[rgb]{0.64,0.00,0.00}{\textbf{#1}}}
\newcommand{\ExtensionTok}[1]{#1}
\newcommand{\FloatTok}[1]{\textcolor[rgb]{0.00,0.00,0.81}{#1}}
\newcommand{\FunctionTok}[1]{\textcolor[rgb]{0.00,0.00,0.00}{#1}}
\newcommand{\ImportTok}[1]{#1}
\newcommand{\InformationTok}[1]{\textcolor[rgb]{0.56,0.35,0.01}{\textbf{\textit{#1}}}}
\newcommand{\KeywordTok}[1]{\textcolor[rgb]{0.13,0.29,0.53}{\textbf{#1}}}
\newcommand{\NormalTok}[1]{#1}
\newcommand{\OperatorTok}[1]{\textcolor[rgb]{0.81,0.36,0.00}{\textbf{#1}}}
\newcommand{\OtherTok}[1]{\textcolor[rgb]{0.56,0.35,0.01}{#1}}
\newcommand{\PreprocessorTok}[1]{\textcolor[rgb]{0.56,0.35,0.01}{\textit{#1}}}
\newcommand{\RegionMarkerTok}[1]{#1}
\newcommand{\SpecialCharTok}[1]{\textcolor[rgb]{0.00,0.00,0.00}{#1}}
\newcommand{\SpecialStringTok}[1]{\textcolor[rgb]{0.31,0.60,0.02}{#1}}
\newcommand{\StringTok}[1]{\textcolor[rgb]{0.31,0.60,0.02}{#1}}
\newcommand{\VariableTok}[1]{\textcolor[rgb]{0.00,0.00,0.00}{#1}}
\newcommand{\VerbatimStringTok}[1]{\textcolor[rgb]{0.31,0.60,0.02}{#1}}
\newcommand{\WarningTok}[1]{\textcolor[rgb]{0.56,0.35,0.01}{\textbf{\textit{#1}}}}
\usepackage{graphicx,grffile}
\makeatletter
\def\maxwidth{\ifdim\Gin@nat@width>\linewidth\linewidth\else\Gin@nat@width\fi}
\def\maxheight{\ifdim\Gin@nat@height>\textheight\textheight\else\Gin@nat@height\fi}
\makeatother
% Scale images if necessary, so that they will not overflow the page
% margins by default, and it is still possible to overwrite the defaults
% using explicit options in \includegraphics[width, height, ...]{}
\setkeys{Gin}{width=\maxwidth,height=\maxheight,keepaspectratio}
% Set default figure placement to htbp
\makeatletter
\def\fps@figure{htbp}
\makeatother
\setlength{\emergencystretch}{3em} % prevent overfull lines
\providecommand{\tightlist}{%
  \setlength{\itemsep}{0pt}\setlength{\parskip}{0pt}}
\setcounter{secnumdepth}{-\maxdimen} % remove section numbering

\title{ELSSP}
\author{}
\date{\vspace{-2.5em}}

\begin{document}
\maketitle

\hypertarget{introduction}{%
\subsection{Introduction}\label{introduction}}

In the United States, 1-2 children are born with hearing loss, per 1,000
births (CDC, 2016
\url{https://www.cdc.gov/ncbddd/hearingloss/2016-data/01-data-summary.html}).
This translates to 114,000 Deaf or Hard of Hearing (DHH) children born
in the U.S. per year (martin et al., 2019). Of these 114,000,
\textasciitilde90\% will be born to hearing parents (Mitchell \&
Karchmer, 2004), in a home where spoken language is likely the dominant
communication method. Depending on the type and degree of hearing loss
and whether the child uses amplification, spoken linguistic input will
be partially or totally inaccessible. Some of these children will
develop spoken language within the range of their hearing peers
(verhaert et al., 2008; geers et al., 2017), but many will face
persistent spoken language deficits (eisenburg, 2007; moeller et al.,
2007; luckner \& cooke, 2010; sarchet et al., 2014), which may later
affect reading ability (kyle \& harris, 2010) and academic achievement
(qi \& mitchell, 2012; karchmer \& mitchell, 2003).

Despite many excellent studies examining language development in DHH
children, there is still a gap in the literature describing and
analyzing spoken language development across the full range of children
receiving state services for hearing loss, with many studies focusing in
on specific subgroups (e.g.~children under age X with Y level of hearing
loss and Z amplification approach, e.g.~yoshinaga-itano et al., 2018;
vohr et al., 2008). In what follows, we first summarize the previous
literature on predictors of spoken language outcomes in DHH children. We
then provide a brief overview of a common vocabulary measure used in the
current study, the MacArthur-Bates Communicative Development Inventory
(CDI). Finally, we turn to an empirical analysis of early vocabulary in
a wide range of young children receiving state services in North
Carolina. We have two broad goals in what follows. First, we aim to
provide a comprehensive description of a heterogeneous group of young
children who receive state services for hearing loss. Second, we aim to
connect the intervention approaches and child characteristics of this
sample with children's vocabulary, with the broader goal of considering
the success of early diagnosis and intervention initiatives.

\hypertarget{predictors-of-language-outcomes}{%
\subsubsection{Predictors of Language
Outcomes}\label{predictors-of-language-outcomes}}

Though the literature points towards spoken language delays and deficits
for DHH children, this is a highly variable population with highly
variable outcomes (pisoni, kronenberger, harris, \& moberly, 2017).
Previous research indicates that gender (Kiese-Himmel \& Ohlwein, 2002;
ching et al., 2013), additional disability (ching et al., 2013;
yoshinaga-itano et al., 2017; verhaert et al., 2008), degree and
configuration of hearing loss (ching et al., 2013; diego lazaro et al.,
2018; vohr et al., 2011; yoshinaga-itano et al., 2017), amplification
(walker et al., 2015), communication (geers et al., 2017), and early
diagnosis/intervention (yoshinaga-itano et al., 2017; 2018) predict
language outcomes in DHH children.

\#\#\#\#Gender For hearing children, the literature points to a female
gender advantage in early language acquisition. Girls speak their first
word earlier (Macoby, 1966), have a larger (bornstein, hahn, \& hayes,
2004; fenson et al., 1994; frank, braginsky, yurovsky, \& marchman,
2017) and faster-growing vocabulary (huttonlocher et al., 1991), and
stronger grammatical and phonological skills (langue, euler, zaretsky,
2016; ozcaliskan \& goldin-meadow, 2010). This finding appears to be
consistent across studies (wallentin, 2009), various spoken languages
(frank, braginsky, marchman, yurovsky, 2019), and gesture (ozcaliskan \&
goldin-meadow, 2010).

The DHH literature presents a more mixed (though rather understudied)
picture. On one hand, DHH girls, like hearing girls, have been found to
have a larger spoken vocabulary than DHH boys (Kiese-Himmel \& Ohlwein,
2002; ching et al., 2013). However, in contrast to their hearing peers,
DHH children do not seem to show a gender-based difference for some
aspects of syntactic development (Pahlevan Nezhad \& Niknehzad, 2014).

\#\#\#\#Comorbidities Additional co-occuring disabilities occur
frequently in the DHH population, perhaps as much as three times more
than in the hearing population (Pollack, 1997). Incidence estimates for
co-occurring disabilities in DHH children range from 25-51\% (Bruce et
al., 2008; Guardino, 2008; Luckner \& Carter, 2001; Picard, 2004;
Schildroth \& Hotto, 1996; Soukup \& Feinstein, 2007; Spencer \&
Marschark, 2010; fortnum \& davis, 1997), with approximately 8\%
children living with 2 or more co-occurring disabilities (schildroth \&
hotto).

Some of these conditions, particularly those which carry risk of
developmental delay (e.g., Down syndrome), result in language delays
independent of hearing loss (chapman, 1997; kristofferson, 2008; weismer
et al., 2010), with cognitive ability more predictive of language
outcomes than presence or absence of a specific disability (meinzen-derr
et al., 2011; sarant et al., 2008). Disability and hearing loss likely
each contribute to a given child's language development (rajput et al.,
2003, van Nierop et al., 2016; ching et al., 2013), with differential
effects of each(vesseur et al., 2016). In some cases, additional
disabilities appear to interact with hearing loss to intensify
developmental delays (birman et al., 2012; pierson et al., 2007).

Furthermore, incidence of hearing loss is higher among children born
premature (defined as \textless{} 37 weeks gestational age). Compared to
an incidence 0.2\% in full-term infants, incidence of hearing loss in
extremely premature infants (defined as \textless33 weeks gestational
age) ranges 2--11\%, with increased prematurity associated with
increased rates of hearing loss (wroblenskla-seniuk et al., 2017).

Independently of hearing status, prematurity is linked to increased risk
of language delay and disorder (costa rechia et al., 2016; van-noort et
al.,2012; carter \& msall, 2017; vohr, 2014; cusson, 2003; barre et al.,
2011). Unfortunately, research on language development in premature DHH
children is scant (Vohr, 2016), so it remains unclear how hearing loss
and prematurity may interact within spoken language skills. One study of
premature infants finds that auditory brainstem response during newborn
hearing screening predicts language performance on the PLS-4 at age 3
(amin et al., 2014), suggesting a link between prematurity and hearing
loss in early childhood, though further research is needed in this
domain.

In extremely premature DHH children, incidence of additional
disabilities may be as high as 73\% (Robertson et al., 2009). Indeed,
pre-term infants with comorbidities have been found to be more likely to
also have hearing loss than those without comorbidities (schmidt et al,
2003), further complicating language development for this population.

\#\#\#\#Audiological characteristics Hearing loss varies in severity,
ranging from slight to profound (clark, 1981). More severe hearing loss
(less access to spoken language) typically results in more difficulty
with spoken language in infancy (vohr et al., 2008), early childhood
(ching et al., 2010, 2013; tomblin et al., 2015; sarant et al., 2008;
siniger et al., 2010) and school-age (wake et al., 2004). Although
profound hearing loss is associated with more pronounced spoken language
difficulty, even mild to moderate hearing loss is associated with
elevated risk of language disorders (delage \& tuller, 2007; blair et al
1985). Hearing loss also varies in whether it affects one ear or both.
Bilateral hearing assists speech perception, sound localization, and
loudness perception in quiet and noisy environments (ching et al.,
2007). The literature on hearing aids and cochlear implants points to
benefits for bilateral auditory input (sarant et al., 2014; lovett et
al., 2010; smulders et al., 2016). At school-age, 3--6\% of children
have unilateral hearing loss (Ross et al., 2010). Although children with
unilateral hearing loss have one ``good ear,'' even mild unilateral
hearing loss has been tied to higher risk of language delays and
educational challenges relative to hearing children (Vila \& Lieu; Lieu,
2013; Lieu, 2004; Lieu, 2012, kiese-himmel). That is, just as in the
bilateral case, more severe hearing loss leads to greater deficits in
language and educational outcomes for children with unilateral hearing
loss (Lieu, 2013; Anne, Lieu, Cohen, 2017). Many DHH children receive
hearing aids (HAs) or cochlear implants (CIs) to boost access to the
aural world. These devices have been associated with better speech
perception and spoken language outcomes (walker et al., 2015; waltzman
et al., 1997; niparko et al., 2010). In turn, aided audibility predicts
lexical abilities with children in HAs (stiles et al., 2012). For both
hearing aids and cochlear implants, earlier fit leads to better spoken
language skills, if the amplification is effective. For hearing aids,
some studies find that children with milder hearing loss who receive
hearing aids earlier have better early language achievement than
children who are fit later (tomblin et al., 2015), but this finding does
not hold for children with severe to profound hearing loss (kiese
himmel, 2002; watkin et al., 2007) (for whom hearing aids are generally
ineffective). Analogously, children who are eligible and receive
cochlear implants earlier have better speech perception and spoken
language outcomes than those implanted later (dettman et al 2007;
miyamoto et al.~2008; svirsky et al, yoshinaga-itano et al., 2018;
artieres et al., 2009; ), with best outcomes for children receiving
implants before their first birthday (dettman et al.~2007).

\#\#\#\#Communication Total Communication (TC) refers to communication
that combines speech, gesture, and elements of sign (but not a full sign
language, such as American Sign Language), sometimes simultaneously.
Clinicians currently employ TC as an alternative or augmentative
communication method for children with a wide range of disabilities
(Branson \& Demchak; Gibbs \& Carswell, 2010; mirenda, 2003).

Compared to total communication, DHH children using an exclusively oral
approach have better speech intelligibility (geers et al., 2017; dillon
et al., 2004; geers et al., 2002; hodges et al., 1999) and auditory
perception (geers et al., 2017; o'donologhue et al., 2000). That said,
there is some debate as to whether an oral approach facilitates higher
spoken language performance, or whether children who demonstrate
aptitude for spoken language are steered towards the oral approach
rather than TC (Hall et al., 2017).

\#\#\#\#1-3-6 Guidelines Early identification (yoshinaga-itano et al.,
2008; yoshinaga-itano et al., 1998; white \& white, 1987; robinshaw,
1995; apuzzo \& yoshinaga-itano, 1995; kennedy et al., 2006) and timely
enrollment in early intervention programs (Vohr et al., 2008, watkin et
al., 2007; ching, 2015; holzinger et al., 2011; vohr et al., 2011) are
associated with better language proficiency. Indeed, DHH children who
receive prompt diagnosis and early access to services have been found to
meet age-appropriate developmental outcomes, including language (stika
et al, 2015).

In line with these findings, the American Academy of Pediatricians (AAP)
has set an initiative for Early Hearing Detection and Intervention
(EHDI). Their EHDI guidelines recommend that DHH children are screened
by 1 month old, diagnosed by 3 months old, and enter early intervention
services by 6 months old. We refer to this guideline as 1-3-6. Meeting
this standard appears to improve spoken language outcomes for children
with HL (yoshinaga-itano et al., 2017; 2018) and the benefits appear
consistent across a range of demographic characteristics.

At a federal level in the U.S., the Early Hearing Detection and
Intervention Act of 2010 (HR. 1246, S. 3199) was passed to develop
state-wide systems for screening, evaluation, diagnosis, and
``appropriate education, audiological, medical interventions for
children identified with hearing loss,'' but policies for early
diagnosis and intervention vary by state. As of 2011, 36 states
(including North Carolina, 15A NCAC 21F .1201 - .1204) mandate universal
newborn hearing screening (national conference of state legislatures,
2011). All states have some form of early intervention programs that
children with hearing loss can access (national association of the
deaf), but these also vary state-by-state. For instance, half of the
states in the US do not consider mild hearing loss an eligibility
criterion for early intervention (holstrum et al., 2008).

In evaluating the success of this initiative, the AAP finds that about
70\% of US children who fail their newborn hearing screening test are
diagnosed with hearing loss before 3 months old, and that 67\% of those
diagnosed (46\% of those that fail newborn hearing screening) begin
early intervention services by 6 months old. These findings suggest that
there may be breaks in the chain from screening to diagnosis and from
diagnosis to intervention, and the effect may be further delays in
language development for children not meeting these guidelines.

\#\#\#Quantifying vocabulary growth in DHH children The MacArthur Bates
Communicative Development Inventory (CDI, fenson et al., 1994) is a
parent-report instrument that gathers information about children's
vocabulary development. The Words and Gestures version of the form
(CDI-WG) is normed for 8-18-month-olds, and includes 398 vocabulary
items that parents indicate whether their child understands or produces,
along with questions about young children's early communicative
milestones. The Words and Sentences version of the form (CDI-WS) is
normed for 16-30-month-olds, and includes 680 vocabulary items that
parents indicate whether their child produces, along with some questions
about grammatical development.. The CDI has been normed on a large set
of participants across many languages (anderson et al., 2002; alcock et
al., 2015; frank et al., 2016; jackson-maldonado et al., 2003).

The CDI has also been validated for DHH children with cochlear implants
(Thal et al., 2009). More specifically, in this validation, researchers
asked parents to complete the CDI, administered the Reynell
Developmental Language Scales, and collected a spontaneous speech
sample. All comparisons between the CDI and the other measures yielded
significant correlations ranging from 0.58 to 0.93. Critically, the
children in this study were above the normed age range for the CDI, and
thus this validation helps to confirm that the CDI is a valid
measurement tool for older DHH children. In further work, Castellanos et
al.~(2016) finds that in children with CIs, number of words produced on
the CDI predicts language, executive function, and academic skills up to
16 years later. Building on this work, several studies have used the CDI
to measure vocabulary development in DHH children (ching et al., 2013;
yoshinaga-itano et al., 2017; yoshinaga-itano et al., 2018; diego lazaro
et al., 2018; vohr et al., 2008; vohr et al., 2011).

\begin{Shaded}
\begin{Highlighting}[]
\CommentTok{# create "Summary of findings of CDI studies in DHH children""}
\end{Highlighting}
\end{Shaded}

\#\#Goals and Predictions

This study aims to 1) characterize the demographic, audiological, and
intervention variability in the population of DHH children receiving
state services for hearing loss; 2) identify predictors of vocabulary
delays; and 3) evaluate the success of early identification and
intervention efforts at a state level. We include two subgroups of DHH
children traditionally excluded from studies of language development:
children with additional disabilities and children with unilateral
hearing loss (e.g., yoshinaga-itano et al., 2018).

For the first and third goal above, we did not have specific hypotheses
and sought to provide descriptive information about a diverse sample of
DHH children receiving state services. For the second, we hypothesized
that male gender, more severe degree of hearing loss, bilateral hearing
loss, no amplification use, prematurity, and presence of additional
disabilities would predict larger spoken vocabulary delay. We did not
have strong predictions regarding communication method, language
background, or presence of other health issues (e.g., congenital heart
malformation).

\#\#Methods Clinical evaluations were obtained through an ongoing
collaboration with the North Carolina Early Language Sensory Support
Program (ELSSP), an early intervention program serving children with
sensory impairments from birth to 36 months. ELSSP passed along
deidentified evaluations to our team after obtaining consent to do so
from each family. No eligibility criteria beyond hearing loss and
receiving an ELSSP evaluation were imposed, given our goal of
characterizing the full range of DHH children with hearing loss in North
Carolina.

The clinical evaluations included demographic and audiological
information, CDI vocabulary scores, and the results of any clinical
assessments administered (e.g., PPVT), all detailed further below. For
some children (n=XXX), multiple evaluations were available from
different timepoints. In these cases, only the first evaluation was
considered for this study due to concerns regarding within-subjects
variance for statistical analysis.

While this collaboration is ongoing, we opted to pause for this analysis
upon receiving data from 100 children. Thus, the reported sample below
consists of XXX children (XXX male/XXX female) ages XXX--XXX(M=XXX,
SD=XXX). Race and SES information was not available. Families were
administered either the WG or WS version of the CDI based on clinician
judgement. Children who were too old for WG, but who were not producing
many words at the time of assessment, were often given WG (n=XXX).
Families for whom Spanish was the primary language (n = XXX) completed
the Spanish version of the CDI (Jackson-Maldonado et al., 2003).

\begin{Shaded}
\begin{Highlighting}[]
\CommentTok{# create "Some table with information about WG/WS breakdown, perhaps average age, average understands score, average produces score"}
\end{Highlighting}
\end{Shaded}

Children in this sample were coded as yes/no for cognitive development
concerns (e.g., Down syndrome, global developmental delays; Cornelia de
Lange syndrome), yes/no for prematurity (i.e., more than 3 weeks
premature), yes/no for health issues (e.g., heart defects, kidney
malformations, VACTERL association), and yes/no for vision loss (not
corrected to normal by surgery or glasses)

\begin{Shaded}
\begin{Highlighting}[]
\CommentTok{# create "Table of different comorbidities, frequency, prematurity"}
\end{Highlighting}
\end{Shaded}

Degree of hearing loss was most often reported with a written
description (e.g., ``mild sloping to moderate'' or ``profound high
frequency loss''). We created 3 variables: hearing loss in the better
ear, hearing loss in the worse ear, and average hearing loss (average of
better and worse ear). Using the ASHA hearing loss guidelines, each of
these was coded with a dB HL value corresponding with the median dB HL
for the level of hearing loss (e.g., moderate hearing loss was coded as
48dB HL), and sloping hearing loss was coded as the average of the
levels (e.g.~mild to moderate was coded as 40.5 dB HL). Participants
were also coded for unilateral or bilateral hearing loss; presence or
absence of Auditory Neuropathy Spectrum Disorder; sensorineural,
conditive, or mixed hearing loss. Amplification was recorded as the
device the child used at the time of assessment--either hearing aid,
cochlear implant, or none.

\begin{Shaded}
\begin{Highlighting}[]
\CommentTok{# create "Table with hearing aid, cochlear implant, age at implantation/fitting, degree, unilateral/bilateral"}
\end{Highlighting}
\end{Shaded}

Communication method was recorded as spoken language, total
communication, or cued speech. One participant had a parent fluent in
sign language, but the reported communication method in the home was
total communication. No child in our sample used sign language.
Participants were also coded as monolingual or multilingual based on
whether families reported using more than one language at home. Total
communication was not counted as multilingualism.

\begin{Shaded}
\begin{Highlighting}[]
\CommentTok{# create "Table with communication method, language background"}
\end{Highlighting}
\end{Shaded}

Age at screening was measured as the child's age in months at their
first hearing screening. Age at screening was only available for (n)
participants. If participants received their newborn hearing screening,
age at screening was recorded as 0 (months). Age at diagnosis was taken
as the age in months when children received their first hearing loss
diagnosis. All children were enrolled in birth-to-three early
intervention services through NC ELSSP, and the date of enrollment was
listed on the clinician evaluation. From the clinician report, we
calculated the number of hours of early intervention services received
per month (including service coordination, speech therapy, and
occupational therapy, among others). Because of the sparse data on
screening age, if participants had an age at diagnosis \(\leq\) 3 mo.
and an age of intervention \(\leq\) 6 mo., they were recorded as meeting
1-3-6. It is possible that a participant did not receive screening by 1
month, but did receive diagnosis by 3 months and services by 6 months.
This special case would be coded as meeting 1-3-6 by our criteria.

\begin{Shaded}
\begin{Highlighting}[]
\CommentTok{# create "Table with meets136 info"}
\end{Highlighting}
\end{Shaded}

\begin{Shaded}
\begin{Highlighting}[]
\CommentTok{# create "Table of variables"}
\end{Highlighting}
\end{Shaded}

Footnotes: Despite exciting, increasing, and converging evidence for
benefits of early sign language exposure (e.g., Schick et al., 2002;
spencer, 1993; davidson et al., 2014; magnuson, 2000; clark et al.,
2016; hratinski \& wilbur, 2016), the majority of DHH children will not
be raised in a sign language environment. This is particularly true for
North Carolina, which does not have a large community of sign language
users, relative to states like Maryland or areas like Washington D.C. or
Rochester, NY. For this reason, we focus on spoken language development.

\end{document}
